%!TEX root = /Users/velrok/Dropbox/TheoInf Seminar/Ausarbeitung/Main.tex

% semantics
\newcommand{\fachwort}[1]{\emph{#1}}

%email format
\newcommand{\emailAdress}[1]{\textless\href{mailto:#1}{#1}\textgreater}

% referenzing
\newcommand{\Abs}[1]{Abschnitt~\ref{#1} }
\newcommand{\Ref}[1]{\ref{#1} }
\newcommand{\Abb}[1]{Abbildung~\ref{fig:#1} }
\newcommand{\Eq}[1]{Gleichung~\eqref{eq:#1} }
\newcommand{\Tab}[1]{Tabelle~\ref{tab:#1} }
\newcommand{\Theo}[1]{Theorem~\eqref{theo:#1} }
\newcommand{\Def}[1]{Definition~\eqref{def:#1} }
\newcommand{\Bsp}[1]{Beispiel~\eqref{bsp:#1} }

\newcommand{\vglHuth}[1]{(vgl. \cite[#1]{huth2004logic}) }
\newcommand{\citeHuth}[1]{\cite[#1]{huth2004logic} }

% special math writings
\newcommand{\Menge}[1]{\mathds{#1}}
\newcommand{\Fancy}[1]{\mathcal{#1}}

% math construction shortcuts
\newcommand{\oneColBox}[1]{
	\boxed{
		\begin{array}{c}
			#1
		\end{array}
	}
}

\newcommand{\oneCol}[1]{
	\begin{array}{c}
		#1
	\end{array}
}

\newcommand{\threeCol}[1]{
	\begin{array}{lll}
		#1
	\end{array}
}

\newcommand{\blackBox}[1]{
	\boxed{
		#1
	}
}

\newcommand{\blueBox}[1]{
	\fcolorbox{blue}{white}{
		#1
	}
}


\newcommand{\dashedBox}[1]{
		\fcolorbox{blue}{white}{
			\begin{array}{c}
				#1
			\end{array}
		}
} 
% \newcommand{\dashedBox}[1]{
% 	\boxed{
% 		\begin{array}{c}
% 			#1
% 		\end{array}
% 	}
% 	\text{make this box dashed}
% }

\newcommand{\FrameDef}{$\Fancy{F} = (W,R)$ }
\newcommand{\ModelDef}{$\Fancy{M} = (W,R,L)$ }
\newcommand{\MMModelDef}{$\Fancy{M} = (W,(R_i)_{i \in \Fancy{A}},L)$ }
\newcommand{\MMFrameDef}{$\Fancy{F} = (W, (R_i)_{i \in \Fancy{A}})$ }
\newcommand{\LabelFuncDef}{$L: W \rightarrow \Fancy{P}(Atome)$ }
\newcommand{\AgentSetDef}{$\Fancy{A} = \{1,2,\dots,n\}$ }

% args: connective condition conclusion
\newcommand{\K}[3]{#1 #2 \wedge #1(#2 \rightarrow #3)\rightarrow #1 #3 }
% args: connective formula
\newcommand{\Truth}[2]{#1 #2 \rightarrow #2 }
% args: connective formula
\newcommand{\Four}[2]{#1 #2 \rightarrow #1 #1 #2 }
% args: connective formula
\newcommand{\Five}[2]{\neg #1 #2 \rightarrow #1 \neg #1 #2 }


% special words
\newcommand{\true}{\emph{Wahr }}
\newcommand{\false}{\emph{Falsch }}

\newcommand{\ML}{Modal-Logik }
\newcommand{\NML}{normale \ML}
\newcommand{\NNML}{nicht-normale \ML}

\newcommand{\deMorganBedingung}{de Morgan Bedingung }
\newcommand{\deMorganRegeln}{de Morgan Regeln }

\newcommand{\KFormel}{$\Box(\phi \rightarrow \psi) \wedge \Box \phi \rightarrow \Box \psi$}
\newcommand{\modelFormel}{$\mathds{M} = (W,R,L)$}

\newcommand{\vierFormel}{$\Box \phi \rightarrow \Box \Box\phi$ }
\newcommand{\fuenfFormel}{$\Diamond \phi \rightarrow \Box \Diamond \phi$ }
\newcommand{\TFormel}{$\Box \phi \rightarrow \phi$ }


\newcommand{\formelSchema}{Formel-Schema }
\newcommand{\formelSchemata}{Formel-Schemata }

\newcommand{\parseTree}{Parse-Tree }

\newcommand{\MML}{Multi-Modal-Logik }
\newcommand{\NMML}{normale \MML}

\newcommand{\MLF}{modal-logische Formel }
\newcommand{\MLFn}{modal-logische Formeln }

\newcommand{\KS}{Kripke-Struktur }

\newcommand{\MAS}{Multi-Agent-Systeme}

\newcommand{\ND}{Natürliche Deduktion }
\newcommand{\AL}{Aussagenlogik }

% table markter shortcurts
\newcommand{\ja}{\checkmark}
\newcommand{\nein}{$\times$}


% für notizen
\usepackage{color}
\newcommand{\todo}[1]{\marginline{\textcolor{red}{\underline{todo}\\#1}}}

\newcommand{\note}[1]{\marginline{\textcolor{blue}{\underline{notiz}\\#1}}}

%fast navigation
\newcommand{\curr}{\label{current}}
