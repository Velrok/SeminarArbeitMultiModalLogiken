%!TEX root = /Users/velrok/Dropbox/TheoInf Seminar/Ausarbeitung/Main.tex

\section{Die Modal-Logik $KT45^n$ (Multi-Agent-Wissen)} % (fold)
\label{sub:the_modal_logic_kt45_n_}


\begin{itemize}
	\item geranlisierung: viele $\Box$
	\item eins für jeden Agenten $\Fancy{A} = \{1,2,\dots,n\}$ 
	\item geschrieben als $K_i$ für Agent $i$ (weil knowlage)
	\item $p,q,r$ für atome
	\item $K_ip$ heißt Agent $i$ weiß $p$
	\item Bsp.: $K_1p \wedge K_1 \neg K_2 K_1 p$ heißt Agent 1 weis K und 1 weis, das 2 nicht weis, dass er p weis
	\item $E_G p$ $G$ ist eine Gruppe von Agenten $G= \{1,2,3,\dots,n\}$
	\item heißt jeder in der Gruppe weis $p$
	\item ist also gleichbedeutent mit $K_1 p \wedge K_2 p \wedge \dots K_n p$
	\item es gelten die selben Bindungsstärken wie in \Abb{bindungsstaerke} beschrieben
	\item wobei $K_i$ als $\Box$ angesehen wird
\end{itemize}

\begin{itemize}
	\item $\phi$ kann bekannter sein als $E_G \phi$ nämlich als $E_G E_G \phi$
	\item bsp: offenes geheimnis: jeder weis es, aber keiner weis das die anderen es wissen
	\item $E_G E_G \phi$ größer $E_G \phi$
	\item und $E_G E_G E_G \phi$ größer als $E_G E_G \phi$
	\item $C_G \phi$ Allgemeinwissen innerhalb $G$ 
	\item weil $E_G E_G \dots \phi$ unendlich, wir aber nur endlich ausdrücken können wird 
	\item $C_G \phi$ als weite Verbindung über die Semantik definiert
	\item $C_G \phi$ heißt als $E_G \phi \wedge E_G E_G \phi \wedge E_G E_G E_G \phi \wedge \dots$
	\item $D_G \phi$ heißt verteiltes Wissen
	\item auch wenn jeder einzelne es nicht weis, wenn sie sich abbrechen kommen sie drauf
	\item D steht für distributed
\end{itemize}



\textbf{multi agent systeme}
\begin{definition}
	\label{def:bnf_kt45n}
	Eine Formel $\psi$ der multi modal Logik $KT45^n$ ist definiert durch folgende Grammatik:
	\begin{equation}
		\label{eqn:bnf_kt45n}
		\phi ::= \bot|\top|p|(\neg\phi)|(\phi\wedge\phi)|(\phi\vee\phi)|(\phi\rightarrow\phi)|
		(\phi\leftrightarrow\phi)|(K_i\psi)|(E_G\psi)|(C_G\psi)|(D_G\psi)
	\end{equation}
	wobei $p$ irgendeine atomare Formel ist und $i \in \Fancy{A}$ sowie $G \subseteq \Fancy{A}$ gilt.
	$E_\Fancy{A}, C_\Fancy{A}, D_\Fancy{A}$ werden zur Einfachheit ohne den extra Index geschrieben $E,C,D$.
	\citeHuth{S.335f}
\end{definition}


Vergleicht man diese Definition mit der von 
\Def{syntax} so stellt man fest, dass anstelle des 
$\Box$ Konnektors es jetzt dien Vielzahl von Konnektoren gibt: 
$K_i, E_G, C_G, D_G$ für alle $ G \subseteq \Fancy{A}$.
Im Folgenden wird gezeigt das sich diese Konnektoren $\Box$ ähnlich verhalten.
Es gibt kein explizites Analogon zu $\Diamond$.
Es wäre aber gleichbedeutend mit $\neg K_i \neg, \neg E_G \neg, \neg C_G \neg, \neg D_G \neg$.


\begin{definition}
	Ein Model \MMModelDef der \MML $KT45^n$ mit der Menge $\Fancy{A}$ von $n$ Agenten wird beschrieben durch drei Dinge:
	\begin{enumerate}
		\item einer Menge $W$ von möglichen Welten
		\item für jedes $i \in \Fancy{A}$, der Gleichheitsrelation $R_i$ auf $W$ $R_i \subseteq W \times W)$ auch Erreichbarkeitsrealtion gennant und
		\item der Labeling-Funktion $L: W \rightarrow \Fancy{P}(Atome)$
	\end{enumerate}
	\cite[S.336f]{huth2004logic}
\end{definition}


\begin{itemize}
	\item jetzt mehre R
	\item eines für jeden Agenten
	\item außerdem muss R eine Gleichheitsrelation sein
	\item Besonderheit in der Datstellung von KT45n Kripkestrukturen
	\begin{itemize}
		\item kanten müssen mit der entsprechenden raltionen beschriftet werden
		\item keine pfeile, weil symetrisch
		\item außerdem werden pfeile auf sich selbst weg gelassen, weil klar ist das alle auf sich selbst zeigen (in jeder der raltionen), weil relfexiv 
	\end{itemize}
\end{itemize}

\todo{Bsp: grphic aus huth S. 336 erstellen}








\begin{definition}
		Gegeben ein Model $\mathds{M} = (W,(R_i)_{i \in \mathds{A}}, L)$ der $KT45^n$ und eine Welt $w \in W$, so definieren $\psi$ als \emph{Wahr} durch die Erfüllung der Relation $x \vDash \psi$ durch folgende Regeln:
		\begin{align}
			x &\Vdash p\text{ gdw. }p \in L(x)\\
			x &\Vdash \neg \phi\text{ gdw. }x \nVdash \phi\\
			%
			x &\Vdash \phi \wedge \psi\text{ gdw. }x \Vdash \phi\text{ und } x \Vdash \psi\\
			x &\Vdash \phi \vee \psi\text{ gdw. }x \Vdash \phi \text{, oder } x \Vdash \psi\\
			%
			x &\Vdash \phi \rightarrow \psi\text{ gdw. }x \Vdash \psi\text{, immer wenn gilt }x \Vdash \phi\\
			%
			x &\Vdash K_i\psi \text{ gdw. } \forall y \in W, R_i(x,y) \text{, } y \Vdash \psi \text{ impliziert}\\
			x &\Vdash E_G\psi \text{ gdw. } \forall i \in G, x \Vdash K_i\psi\\
			x &\Vdash C_G\psi \text{ gdw. } \forall k \geq 1 \text{, und es gilt } x \Vdash E^k_G\psi \text{.} \text{Wobei } E^k_G \text{ meint } E_{G}E_{G}\dots E_{G} \text{ k-mal.}\\
			x &\Vdash D_G\psi \text{ gdw. } \forall y \in W, y \Vdash \psi \text{gilt, immer wenn auch } R_i(x,y), \forall i \in G \text{gilt.}\\
		\end{align}
		\cite[S.337]{huth2004logic}
\end{definition}


\paragraph{Valide Formeln in $KT45^n$}



\subsection{\ND in $KT45^n$}

\subsection{Formalisierung des Wise Men Rätsels in $KT45^n$}










% subsection the_model_logic_kt45_n_ (end)
