%!TEX root = /Users/velrok/Dropbox/TheoInf Seminar/Ausarbeitung/Main.tex


\section{Attribute einer Modal-Logik} % (fold)
\label{sub:attribute_einer_modal_logik}
Die Attribute einer \ML werden dadurch bestimmt welche Formeln als valide vorausgesetzt werden.
Alle \NML setzen die Validität der Formel $K$, die Formel für die Logische Konsequentz \todo{Quelle referenzieren}, voraus.
\NNML sind nicht Teil dieser Arbeit. Der interessierte Leser sein an Priest \cite[S.75ff]{Priest:2008} verwiesen.
Will man eine eigene \ML kreieren ist es wichtig sich genaue Gedanken darüber zu machen, welcher Formeln die als valide voraussetzt, weil das die Eigenschaften der zu modellierenden Wahrheit bestimmt.
Im Folgenden wird erst das für \NML zwingend erforderliche Basisattribut $K$ vorgestellt und danach auf die anderen optionalen Eigenschaften eingegangen.
Zum Schluss werden ein paar \NML und deren Eigenschaften beschrieben und erklärt warum die gewählten Eigenschaften für die Modalität wünschenswert / wichtig / notwendig sind.\\
%
\curr
\begin{definition}
	\label{def:substitution}
	Sei $\mathds{L}$ eine Menge von Formel-Schemata der Modal Logik und $\Gamma \cup {\psi}$ eine Menge von modal logischen Formeln.
	
	\begin{itemize}
		\item Die Menge $\Gamma$ ist abgeschlossen gegenüber der Substitution von Instanzen gdw.  $\psi \in \Gamma$. 
		Dann gilt auch, dass jede Substitutionsinstanz von $\psi$ auch in $\Gamma$ ist.
		\item Sei $\mathds{L}_c$ die kleinste Menge die alle Instanzen von $\mathds{L}$ enthält.
		\item Aus $\Gamma$ folgt semantisch $\psi$ in $\mathds{L}$ gdw. alle Modelle, deren Frame $\mathds{L}$ erfüllt und alle Welten $x$ in diesem Modell, $x$ erfüllt $\Gamma$, gilt.
		In diesem Fall sagen wir $\Gamma \vDash_{\mathds{L}} \psi$ ist erfüllt.
	\end{itemize}
	\cite[S.326]{huth2004logic}
\end{definition}


\paragraph{Das Basis Atribute $K$} % (fold)
\label{par:das_basis_atribute_k_} Die Formel $K$ \KFormel besagt, dass es nur normale Welten gibt. Das heißt die Wahrheitsmodulation $\Box$ verhält sich immer gleich im Gegensatz zu \NNML wo es \fachwort{nicht normale} Welten geben kann.
In nicht normalen Welten ist vereinfacht formuliert alles möglich und nichts notwendig.\cite[S.75]{Priest:2008}



\paragraph{Weitere Attribute} % (fold)
\label{par:weitere_attribute} 

Neben der Formel für $K$ gibt es weitere typische Formeln die, sofern sie als valide vorausgesetzt werden einer \NML gewissen Eigenschaften verleihen.


\todo{weiter Atribute diskutieren}
\textbf{einige Modal Logiken}



\begin{definition}
	Ein Modell der intuitionistischen Aussagenlogik ist ein Model \modelFormel der Logik $KT45$, sodass $R(x,y)$ immer $L(x) \subseteq L(y)$ impliziert.
	Gegeben einer modal logische Formel nach \eqref{eqn:bnf}, definieren wir $x \Vdash \psi$ wie in Definition \eqref{def:reasoning} mit Außnahme der Reglen für $\rightarrow$ und $\neg$.
	\begin{itemize}
		\item $\psi \rightarrow \phi$ definierten wir als $x \Vdash \psi \rightarrow \phi$ gdw. $\forall y R(x,y)$ auch $y \Vdash \phi$ gilt, immer wenn $y \Vdash \psi$ gilt.
		\item $\neg \psi$ definierten wir als $x \Vdash \neg \psi$ gdw. $\forall y R(x,y)$ $y \nVdash \psi$ der Fall ist.
	\end{itemize}
\end{definition}
\cite[S.328]{huth2004logic}

\begin{definition}
	Gegeben einer Menge von Formel Schemata $\mathds{L}$.
	Definieren wir $\Gamma \Vdash_\mathds{L} \psi$ als valide, wenn es einen Beweis im natürlichen Folgerungssystem für modal Logiken gibt, der um die Axiome in $\mathds{L}$ und den Annahmen in $\Gamma$ erweitert ist.
	\todo{bessere Forumlierung finden}
	\todo{natural deduction system beser übersetzen}
\end{definition}
\cite[S.330]{huth2004logic}




% paragraph abgesehen_davon_gibt_es_weitere_attribute (end)
% paragraph das_basis_atribute_k_ (end)
% subsection attribute_einer_modal_logik (end)
