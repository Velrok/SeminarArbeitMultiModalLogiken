%!TEX root = /Users/velrok/Dropbox/TheoInf Seminar/Ausarbeitung/Main.tex


\section{Attribute einer Modal-Logik} % (fold)
\label{sub:attribute_einer_modal_logik}
Die Attribute einer \ML werden dadurch bestimmt welche Formeln als valide vorausgesetzt werden.
Alle \NML setzen die Validität der Formel $K$, die Formel für die Logische Konsequentz \todo{Quelle referenzieren}, voraus.
\NNML sind nicht Teil dieser Arbeit. Der interessierte Leser sein an Priest \cite[S.75ff]{Priest:2008} verwiesen.
Will man eine eigene \ML kreieren ist es wichtig sich genaue Gedanken darüber zu machen, welcher Formeln die als valide voraussetzt, weil das die Eigenschaften der zu modellierenden Wahrheit bestimmt.
Im Folgenden wird erst das für \NML zwingend erforderliche Basisattribut $K$ vorgestellt und danach auf die anderen optionalen Eigenschaften eingegangen.
Zum Schluss werden ein paar \NML und deren Eigenschaften beschrieben und erklärt warum die gewählten Eigenschaften für die Modalität wünschenswert / wichtig / notwendig sind.\\
%
\paragraph{Das Basis Atribute $K$} % (fold)
\label{par:das_basis_atribute_k_} Die Formel $K$ \KFormel besagt, dass es nur normale Welten gibt. Das heißt die Wahrheitsmodulation $\Box$ verhält sich immer gleich im Gegensatz zu \NNML wo es \fachwort{nicht normale} Welten geben kann.
In nicht normalen Welten ist vereinfacht formuliert alles möglich und nichts notwendig.\cite[S.75]{Priest:2008}


\paragraph{Weitere Attribute} % (fold)
\label{par:weitere_attribute} 

Neben der Formel für $K$ gibt es weitere typische Formeln die, sofern sie als valide vorausgesetzt werden einer \NML gewissen Eigenschaften verleihen.\\
\\
%
\begin{table}
	\label{tab:attributes}
	\centering
	\begin{tabular}{cl}
	\hline
	\hline
	Name & Formel Schema\\
	\hline
	T & $\Box \phi \rightarrow \phi$\\
	B & $\phi \rightarrow \Box \Diamond\phi$\\
	D & $\Box \phi \rightarrow \Diamond \phi$\\
	4 & $\Box \phi \rightarrow \Box \Box \phi$\\
	5 & $\Diamond \phi \rightarrow \Box \Diamond \phi$\\
	\hline
	\end{tabular}\\
	\caption{Attribut Bezeichnungen und entsprechende \formelSchemata}
\end{table}
\\
Will man z.B.: eine Logik für die Notwendigkeit erstellen, so will man das die Formel $T$ $\Box \phi \rightarrow \phi$ für alle Welten zutrifft.
Den etwas das notwendigerweise \true ist sollte auch einfach \true sein.
Das selbe gilt für Wissen: Wenn man etwas weis, dann ist das auch wahr oder anders man hat kein falsches Wissen.
Das mag nicht der Realität entsprechen, ist aber eine Idealisierung die man für Multi-Agent-System i.d.R. haben möchte.
Will man hingegen die Wahrheitsmodalität Glauben modellieren, so wäre es unklug $T$ aufzunehmen.
Der Glaube zeichnet sich nämlich dadurch aus, das man auch Dinge glauben kann die \false sind.\\
\todo{Professioneller formulieren}
\\
\begin{table}
	\label{tab:wahrheitsModsUndAttr}
	\centering
		\begin{tabular}{lccccc}
		$\Box \phi$ & 
		\begin{sideways}
			 $\Box \phi \rightarrow \phi$
		\end{sideways} & 
		\begin{sideways}
			$\phi \rightarrow \Box \Diamond\phi$
		\end{sideways} & 
		\begin{sideways}
			$\Box \phi \rightarrow \Diamond \phi$
		\end{sideways} &
		\begin{sideways}
			 $\Box \phi \rightarrow \Box \Box \phi$
		\end{sideways} &
		\begin{sideways}
			 $\Diamond \phi \rightarrow \Box \Diamond \phi$
		\end{sideways}\\
		\hline

		Es ist notwendig, dass 				& \ja   & \ja 	& \ja 	& \ja		& \ja 	\\
		Es wird immer wahr sein, dass & \nein	& \ja		& \nein & \ja 	& \nein	\\
		Es ist sollte sein, dass 			& \nein & \nein & \ja 	& \nein & \nein	\\
		Agent Q glaubt, dass	 				& \nein & \ja 	& \ja 	& \ja 	& \ja		\\
		Agent Q weis, dass 						& \ja 	& \ja 	& \ja 	& \ja 	& \ja		\\
		\hline
		\hline
		\end{tabular}\\
				\caption{Verschiedenen Modulationen von Wahrheit und ihre Eigenschaften}
\end{table}\\
\\




\todo{weiter Atribute diskutieren}








% paragraph abgesehen_davon_gibt_es_weitere_attribute (end)
% paragraph das_basis_atribute_k_ (end)
% subsection attribute_einer_modal_logik (end)
