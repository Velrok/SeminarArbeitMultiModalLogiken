\section{Intuistische \ML} % (fold)
\label{sec:intuistische_ml}
\todo{mehr schreiben}

\begin{definition}
	\label{intuistischeLogiken}
	Ein Modell der intuitionistischen Aussagenlogik ist ein Model \modelFormel der Logik $KT45$, sodass $R(x,y)$ immer $L(x) \subseteq L(y)$ impliziert.
	Gegeben einer modal logische Formel nach \eqref{eqn:bnf}, definieren wir $x \Vdash \psi$ wie in Definition \eqref{def:reasoning} mit Außnahme der Reglen für $\rightarrow$ und $\neg$.
	\begin{itemize}
		\item $\psi \rightarrow \phi$ definierten wir als $x \Vdash \psi \rightarrow \phi$ gdw. $\forall y R(x,y)$ auch $y \Vdash \phi$ gilt, immer wenn $y \Vdash \psi$ gilt.
		\item $\neg \psi$ definierten wir als $x \Vdash \neg \psi$ gdw. $\forall y R(x,y)$ $y \nVdash \psi$ der Fall ist.
	\end{itemize}
\end{definition}
\cite[S.328]{huth2004logic}


% section intuistische_ml (end)