%!TEX root = /Users/velrok/Dropbox/TheoInf Seminar/Ausarbeitung/Main.tex


\section{Das Wise-Men Rätsel} % (fold)
\label{sub:das_wise_men_raetsel}

Das \emph{Wise-Men-Puzzle} ist ein klassisches Beispiel dafür wie ein Agent aufgrund von Allgemeinwissen und das Wissen über das Wissen oder Unwissen andere Folgerungen ziehen kann.

\begin{puzzle}
	\label{puz:wiseMen}
	Es gibt 3 weise Männer.
	Es gehört zum Allgemeinwissen - etwas das jeder weis, und jeder weis, dass es jeder weis, was wiederum jeder weis usw. -, dass es 3 rote und 2 weise Hüte gibt.
	Der König setzt jedem der weisen Männer einen Hut auf, sodass jeder nur die Hüter der anderen, nicht jedoch seinen eigenen sehen kann.
	Danach fragt er der Reihe nach jeden ob er weis welche Farbe sein Hut hat.
	Gehen wir davon aus, das sowohl der Erste als auch der Zweite es nicht weis, dann folgt daraus, dass der Dritte Wissen muss welche Farbe sein Hut hat.
	\\
	Warum?\\
	Welche Farbe hat sein Hut?
\end{puzzle}
%
%
Das Rätsel setzt folgendes Vorraus:
\begin{itemize}
	\item Alle Beteiligten sind ehrlich
	\item Alle Beteiligten sind schlau (übersehen keine Folgerungen)
	\item Alle Beteiligten wissen das die anderen schlau sind
	\item Alle Beteiligten besitzen das selbe Allgemeinwissen
\end{itemize}
%
%
Im folgenden wird das Rätsel umgangssprachlich und durch Überlegungen gelöst. In \Abs{sub:the_modal_logic_kt45_n_} wird das Rätsel in der \MML $KT45^n$ formalisiert und formal gefolgert.\\
\\
Beginnen wir damit alle möglichen Kombinationen zu notieren:\\
%
\begin{tabular}{ccc}
\texttt{R R R} &   & \texttt{W R R}\\
\texttt{R R W} &   & \texttt{W R W}\\
\texttt{R W R} &   & \texttt{W W R}\\
\texttt{R W W} &   &   \\
\end{tabular}
%
Wobei die Notation \texttt{R R W} bezeichnet, dass der Erste und Zweite einen roten und der Dritte einen weißen Hut tragen.
Der Fall \texttt{W W W} kann nicht auftreten, weil es keine 3 weißen Hüte gibt.\\
Betrachten wir das Rätsel mal aus der Perspektive des 2. und 3. Weisen. 
Nach der negativ Aussage vom Ersten kann der Zweite folgern, dass \texttt{R W W}, nicht der Fall ist, sonst wüste der 1. das er einen roten Hut trägt. 
Mit der selben Argumentation kann der 3. den Fall \texttt{W R W} ausschließen. 
Damit bleiben folgende Kombinationen:\\
\begin{tabular}{ccc}
\texttt{R R R} &   & \texttt{W R R}\\
\texttt{R R W} &   & \sout{\texttt{W R W}}\\
\texttt{R W R} &   & \texttt{W W R}\\
\sout{\texttt{R W W}} &   &   \\
\end{tabular}
%
Der 3. kann außerdem den Fall \texttt{R R W} ausschließen, denn wäre dies der Fall gewesen hätte der 2. gefolgert, dass es einer der beiden Kombinationen \texttt{R R W} oder \texttt{R W W} zutreffen muss.
Der Fall \texttt{R W W} konnte aber schon durch die Aussage des Ersten ausgeschlossen werden.
Wäre also \texttt{R R W} der Fall gewesen, so hätte der Zwei gewusst, dass er einen roten trägt. Da er das aber nicht sagt, kann dieser Fall ausgeschlossen werden.
Damit bleiben übrig:\\
%
\begin{tabular}{ccc}
\texttt{R R R} &   & \texttt{W R R}\\
\sout{\texttt{R R W}} &   & \sout{\texttt{W R W}}\\
\texttt{R W R} &   & \texttt{W W R}\\
\sout{\texttt{R W W}} &   &   \\
\end{tabular}
Wie man sehen kann, trägt der 3. in jedem der Fällt einen roten Hut.
Deswegen kann er folgern, dass er einen roten Hut aufhaben muss, weil sonst einer der anderen anders geantwortet hätte.\\
Das zeigt, warum es notwendig ist, das alle Beteiligten schlau sind, nichts übersehen, nicht lügen und all dies zum Allgemeinwissen der Beteiligten zählt.




% subsection das_wise_men_ (end)
