%!TEX root = /Users/velrok/Dropbox/TheoInf Seminar/Ausarbeitung/Main.tex

\section{Ähnlichkeitstheorie} % (fold)
\label{sub:Aehnlichkeitstheorie}

Die Ähnlichkeitstheorie besagt, das sich die Eigenschaften einer Modal-Logik in der Relation der entsprechenden Kripkestrucktur widerspiegeln und vis versa. 
Dies schafft einen neuen Zugang zum Design von Modal-Logiken. 
In manchen Fällen mag es einfacher sein in den notwendigen Eigenschaften in Form von Formel-Schemata zu denken, in anderen ist es evtl. einfacher das Problem über die Relation zu verstehen. 
Im Folgenden wird gezeigt, wie die einzelnen Eigenschaften mit der Kripke-Struktur-Relation zusammen hängen.

\begin{definition}
	\label{def:frame}
	Ein Frame $F = (W,R)$ ist eine Menge von Welten $W$ und eine binäre Relation $R$ auf $W$.
\end{definition}
\cite[S.322]{huth2004logic}
\note{labeling funktion fehlt}


\begin{definition}
	\label{def:frame_erfuellt}
	Ein Frame $F$ erfüllt eine modal logische Formal $\psi$, wenn für jede \fachwort{Labelfunktion} $L: W \rightarrow P(Atome)$ und jedes $w \in W$, es der Fall ist, dass $M,w \vDash \psi$ gilt. $M$ ist das Model: $M = (W,R,L)$.
	In diesem Falle schreiben wir $F \vDash \psi$.
\end{definition}
\cite[S.322f]{huth2004logic}
\todo{entsprechende Buchstaben in MathDS auszeichnen}





\todo{mehr schreiben}

% subsection Aehnlichkeitstheorie (end)
