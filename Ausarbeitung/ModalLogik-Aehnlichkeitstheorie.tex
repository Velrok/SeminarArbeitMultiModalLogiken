%!TEX root = /Users/velrok/Dropbox/TheoInf Seminar/Ausarbeitung/Main.tex
\section{Ähnlichkeitstheorie} % (fold)
\label{sub:Aehnlichkeitstheorie}

Die Ähnlichkeitstheorie besagt, das sich die Eigenschaften einer Modal-Logik in der Relation der entsprechenden Kripkestrucktur widerspiegeln und vis versa. 
Dies schafft einen neuen Zugang zum Design von Modal-Logiken. 
In manchen Fällen mag es einfacher sein in den notwendigen Eigenschaften in Form von Formel-Schemata zu denken, in anderen ist es evtl. einfacher das Problem über die Relation zu verstehen. 
Im Folgenden wird gezeigt, wie die einzelnen Eigenschaften mit der Kripke-Struktur-Relation zusammen hängen.


\paragraph{Zusammenhang zwischen der Relation $R$ und den validen \formelSchemata}

Zusamenhang bisher auf basis von Intuition:



\textbf{About Frames}

Was Frames sind:
\begin{definition}
	\label{def:frame}
	Ein Frame $F = (W,R)$ ist eine Menge von Welten $W$ und eine binäre Relation $R$ auf $W$.
\end{definition}
\cite[S.322]{huth2004logic}
\note{labeling funktion fehlt}

Wozu Frames gut sind:


Wann ein Frame etwas erfüllt:
\begin{definition}
	\label{def:frame_erfuellt}
	Ein Frame $F$ erfüllt eine modal logische Formal $\psi$, wenn für jede \fachwort{Labelfunktion} $L: W \rightarrow P(Atome)$ und jedes $w \in W$, es der Fall ist, dass $M,w \vDash \psi$ gilt. $M$ ist das Model: $M = (W,R,L)$.
	In diesem Falle schreiben wir $\mathds{F} \vDash \psi$.
	\cite[S.322f]{huth2004logic}
\end{definition}

Wenn ein Frame eine Formel erfüllt erfüllt es auch das entsprechende Schema.


Beispiel dafür (5.12)

Theorem warum das Bsp. Frame \TFormel erfüllt aber \vierFormel nicht:

Beweis dafür:

Behauptung das Tabelle s.u. gilt

<Tabelle über name schema und R Eigenschaften>











\begin{itemize}
	\item Was genau ist das? : Ein bewiesenes Theorem.
	\item Wie ist das begründet? : Ursprung Intuition, kann aber bewiesen werden.
	\item Wie werden die Definitionen genutzt? : Frames sind eine veralgemeinerung von Modellen. Können analog zu formel Schemata in Formeln gesehen werden.
	\item Welche Behauptungen werden aufgestellt? : siehe tabelle 5.12
	\item Wie werden diese Bewiesen? : Ringschluss S. 325
	\item Was sind Frames? : die analogie zu Formel Schemeta. \KM ohne Labels
	\item Wozu brauch man Frames? : für die Beweise der Theoreme. Um zu zeigen, dass ein Model eine Eigenschaft hat die unabhängig ist von der Konkreten Wissensbasis -> alg. gültig.
\end{itemize}











\todo{mehr schreiben}

% subsection Aehnlichkeitstheorie (end)
