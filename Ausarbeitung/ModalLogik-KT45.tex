%!TEX root = /Users/velrok/Dropbox/TheoInf Seminar/Ausarbeitung/Main.tex


\section{Die Modal-Logik $KT45$ (Wissen)} % (fold)
\label{sub:the_normal_modal_logic_s5_}


\paragraph{weitere Modal Logiken} % (fold)
\label{par:weitere_modal_logiken}

% paragraph weitere_modal_logiken (end)

\begin{definition}
	Ein Modell der intuitionistischen Aussagenlogik ist ein Model \modelFormel der Logik $KT45$, sodass $R(x,y)$ immer $L(x) \subseteq L(y)$ impliziert.
	Gegeben einer modal logische Formel nach \eqref{eqn:bnf}, definieren wir $x \Vdash \psi$ wie in Definition \eqref{def:reasoning} mit Außnahme der Reglen für $\rightarrow$ und $\neg$.
	\begin{itemize}
		\item $\psi \rightarrow \phi$ definierten wir als $x \Vdash \psi \rightarrow \phi$ gdw. $\forall y R(x,y)$ auch $y \Vdash \phi$ gilt, immer wenn $y \Vdash \psi$ gilt.
		\item $\neg \psi$ definierten wir als $x \Vdash \neg \psi$ gdw. $\forall y R(x,y)$ $y \nVdash \psi$ der Fall ist.
	\end{itemize}
\end{definition}
\cite[S.328]{huth2004logic}

\begin{definition}
	Gegeben einer Menge von Formel Schemata $\mathds{L}$.
	Definieren wir $\Gamma \Vdash_\mathds{L} \psi$ als valide, wenn es einen Beweis im natürlichen Folgerungssystem für modal Logiken gibt, der um die Axiome in $\mathds{L}$ und den Annahmen in $\Gamma$ erweitert ist.
	\todo{bessere Forumlierung finden}
	\todo{natural deduction system beser übersetzen}
\end{definition}
\cite[S.330]{huth2004logic}

\begin{definition}
	\label{def:substitution}
	Sei $\mathds{L}$ eine Menge von Formel-Schemata der Modal Logik und $\Gamma \cup {\psi}$ eine Menge von modal logischen Formeln.
	
	\begin{itemize}
		\item Die Menge $\Gamma$ ist abgeschlossen gegenüber der Substitution von Instanzen gdw.  $\psi \in \Gamma$. 
		Dann gilt auch, dass jede Substitutionsinstanz von $\psi$ auch in $\Gamma$ ist.
		\item Sei $\mathds{L}_c$ die kleinste Menge die alle Instanzen von $\mathds{L}$ enthält.
		\item Aus $\Gamma$ folgt semantisch $\psi$ in $\mathds{L}$ gdw. alle Modelle, deren Frame $\mathds{L}$ erfüllt und alle Welten $x$ in diesem Modell, $x$ erfüllt $\Gamma$, gilt.
		In diesem Fall sagen wir $\Gamma \vDash_{\mathds{L}} \psi$ ist erfüllt.
	\end{itemize}
	\cite[S.326]{huth2004logic}
\end{definition}

% subsection the_normal_modal_logic_s5_ (end)