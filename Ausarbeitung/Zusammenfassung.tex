%!TEX root = /Users/velrok/Dropbox/TheoInf Seminar/Ausarbeitung/Main.tex

\chapter{Zusammenfassung} % (fold)
\label{sec:zusammenfassung}

\begin{itemize}
	\item \ML mehre Modalitäten von Wahrheit
	\item Notwendigkeit $\Box$ und Möglichkeit $\Diamond$
	\item Attribute einer \ML besonders $K,T,4,5$
	\item modellierung durch Possible worlds (Kripke)
	\item Ähnlichkeitstheorie = Zusammenhand zwischen Attributen der Logik und 
	\item Eigenschaften der Kripke Relation
	\item Beispiel der Knowlage Logik $KT45$ (Folgerungen über Wissen)
	\item \MML als \ML mit mehr als einem $\Box$ Operator
	\item Beispiel der $KT45^n$
	\item Behandlung auf Basis des \emph{wise men} Rätsels
	\item \ND als Rechensystem für \AL \ML und \MML
	\item jeweils neue Regeln
	\item Auswirkungen der Attribute $T,4,5$ auf den Regelsatz
	\item Am Ende Formalisierung und Beweis des \emph{wise men} Rätsels durch Verwendung der \ND in $KT45^n$
\end{itemize}


% section zusammenfassung (end)
