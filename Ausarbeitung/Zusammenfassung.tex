%!TEX root = /Users/velrok/Dropbox/TheoInf Seminar/Ausarbeitung/Main.tex

\chapter{Zusammenfassung} % (fold)
\label{sec:zusammenfassung}

Die \ML beschreibt eine Modalitäten von Wahrheit,
als Beispiel wurden Notwendigkeit und Wissen aufgeführt.
Die Notwendigkeit wird in der \ML mit dem Symbol $\Box$ und die Möglichkeit mit $\Diamond$ notiert.\\
Dabei kann eine \ML verschiedene Attribute besitzen.
Die Attribute $K, T, 4$ und $5$ wurden besondern ausführlich behandelt u.a. weil sie für die Modalität Wissen, auf der der Fokus der Arbeit liegt, wichtig sind.\\
\ML werden durch den \emph{possible world} Ansatz von Kripke modelliert.
Unter Verwendung von Kripke Strukturen ergeben sich Graphen in eine Welt mit anderen möglichen Welten durch die $R$ Relation verbunden werden.\\
Die Ähnlichkeittheorie belegt einen mathematischen Zusammenhang zwischen der Attributen einer \ML und den Eigenschaften der entsprechenden Kripke Relation $R$.\\
Als ein Beispiel für eine \ML wurde die Logik $KT45$ für Wissen genauer untersucht und beschrieben.\\
\MML unterscheiden sich von \ML durch ihre Vielzahl an $\Box$ Operatoren.
Ein Beispiel dafür war die \MML $KT45^n$.
Anhand des Beispiels \emph{wise men} wurde gezeigt, wie ein umgangssprachlich formuliertes Problem formalisiert und formal gefolgert werden kann.\\
Dafür wurde die \ND eingesetzt.
Die \ND ist ein Rechensystem für \AL, \ML und \MML Logik, mit einem jeweils angepassten Regelsatz für die spezielle Logik in der gearbeitet wird.
Attribute wie $K, T, 4$ und $5$ haben dabei direkte Auswirkungen auf diesen Regelsatz.


% section zusammenfassung (end)
