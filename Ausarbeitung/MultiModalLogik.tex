%!TEX root = /Users/velrok/Dropbox/TheoInf Seminar/Ausarbeitung/Main.tex


\chapter{Multi-Modal-Logic} % (fold)
\label{sec:multi_modal_logic}
\MML sind \NML die mehr als eine Modularität der Wahrheit enthalten.
Ein Beispiel dafür sind Multi-Agent-Systeme. 
In einem solchen System, kann ein Agent nicht nur Folgerungen auf Basis seines Wissens, sonder auch auf Basis des Wissen über das Wissen der anderen anstellen. Also Aussagen der Art: Weil \emph{ich} weis das \emph{er} \textbf{A} weis kann ich \textbf{B} folgern.
Konkret wird dieses Kapitel die Logik $KT45^n$, den allgemeinen Fall der Wissenslogik $KT45$, anhand der klassischen Logik Rätsel \emph{Wise-Men} und \emph{Muddy-Children} darstellen.

%!TEX root = /Users/velrok/Dropbox/TheoInf Seminar/Ausarbeitung/Main.tex


\section{Das Wise-Men Rätsel} % (fold)
\label{sub:das_wise_men_raetsel}

Das \emph{Wise-Men-Puzzle} ist ein klassisches Beispiel dafür wie ein Agent aufgrund von Allgemeinwissen und das Wissen über das Wissen oder Unwissen andere Folgerungen ziehen kann.

\begin{puzzle}
	\label{puz:wiseMen}
	Es gibt 3 weise Männer.
	Es gehört zum Allgemeinwissen - etwas das jeder weiß, und jeder weiß, dass es jeder weiß, was wiederum jeder weiß usw. -, dass es 3 rote und 2 weiße Hüte gibt.
	Der König setzt jedem der weisen Männer einen Hut auf, sodass jeder nur die Hüter der anderen, nicht jedoch seinen eigenen sehen kann.
	Danach fragt er der Reihe nach jeden ob er weiß welche Farbe sein Hut hat.
	Gehen wir davon aus, dass sowohl der Erste als auch der Zweite es nicht weiß, dann folgt daraus, dass der Dritte Wissen muss welche Farbe sein Hut hat.
	\\
	Warum?\\
	Welche Farbe hat sein Hut?
\end{puzzle}
%
%
Das Rätsel setzt folgendes Vorraus:
\begin{itemize}
	\item Alle Beteiligten sind ehrlich
	\item Alle Beteiligten sind schlau (übersehen keine Folgerungen)
	\item Alle Beteiligten wissen das die anderen schlau sind
	\item Alle Beteiligten besitzen dasselbe Allgemeinwissen
\end{itemize}
%
%
Im folgenden wird das Rätsel umgangssprachlich und durch Überlegungen gelöst. In \Abs{sub:the_modal_logic_kt45_n_} wird das Rätsel in der \MML $KT45^n$ formalisiert und formal gefolgert.\\
\\
Beginnen wir damit alle möglichen Kombinationen zu notieren:\\
%
\begin{tabular}{ccc}
\texttt{R R R} &   & \texttt{W R R}\\
\texttt{R R W} &   & \texttt{W R W}\\
\texttt{R W R} &   & \texttt{W W R}\\
\texttt{R W W} &   &   \\
\end{tabular}
%
Wobei die Notation \texttt{R R W} bezeichnet, dass der Erste und Zweite einen roten und der Dritte einen weißen Hut tragen.
Der Fall \texttt{W W W} kann nicht auftreten, weil es keine 3 weißen Hüte gibt.\\
Betrachtet man das Rätsel mal aus der Perspektive des 2. und 3. Weisen. 
Nach der negativ Aussage vom Ersten kann der Zweite folgern, dass \texttt{R W W}, nicht der Fall ist, sonst wüste der 1. das er einen roten Hut trägt. 
Mit der selben Argumentation kann der 3. den Fall \texttt{W R W} ausschließen. 
Damit bleiben folgende Kombinationen:\\
\begin{tabular}{ccc}
\texttt{R R R} &   & \texttt{W R R}\\
\texttt{R R W} &   & \sout{\texttt{W R W}}\\
\texttt{R W R} &   & \texttt{W W R}\\
\sout{\texttt{R W W}} &   &   \\
\end{tabular}
%
Der 3. kann außerdem den Fall \texttt{R R W} ausschließen, denn wäre dies der Fall gewesen hätte der 2. gefolgert, dass es einer der beiden Kombinationen \texttt{R R W} oder \texttt{R W W} zutreffen muss.
Der Fall \texttt{R W W} konnte aber schon durch die Aussage des Ersten ausgeschlossen werden.
Wäre also \texttt{R R W} der Fall gewesen, so hätte der Zwei gewusst, dass er einen roten trägt. Da er das aber nicht sagt, kann dieser Fall ausgeschlossen werden.
Damit bleiben übrig:\\
%
\begin{tabular}{ccc}
\texttt{R R R} &   & \texttt{W R R}\\
\sout{\texttt{R R W}} &   & \sout{\texttt{W R W}}\\
\texttt{R W R} &   & \texttt{W W R}\\
\sout{\texttt{R W W}} &   &   \\
\end{tabular}
Wie man sehen kann, trägt der 3. in jedem der Fällt einen roten Hut.
Deswegen kann er folgern, dass er einen roten Hut aufhaben muss, weil sonst einer der anderen anders geantwortet hätte.\\
Das zeigt, warum es notwendig ist, dass alle Beteiligten schlau sind, nichts übersehen, nicht lügen und all dies zum Allgemeinwissen der Beteiligten zählt.




% subsection das_wise_men_ (end)

%!TEX root = /Users/velrok/Dropbox/TheoInf Seminar/Ausarbeitung/Main.tex


\subsection{Das Muddy-Children Rätsel} % (fold)
\label{sub:das_muddy_children_raetsel}

% section das_muddy_children_rätsel (end)

%!TEX root = /Users/velrok/Dropbox/TheoInf Seminar/Ausarbeitung/Main.tex

\section{Die Modal-Logik $KT45^n$ (Multi-Agent-Wissen)} % (fold)
\label{sub:the_modal_logic_kt45_n_}

Die \MML $Kt45^n$ ist eine Verallgemeinerung der \ML $KT45$, in der Hinsicht als dass es mehrerer $\Box$ ähnliche Operatoren gibt.
Jeder Agent einer Menge \AgentSetDef hat seine eigene Relation $R_i$ auf $W$.
Der Operator eines Agenten $i$ wird notiert als $K_i$ mit $K$ für \emph{Knowlage} (Wissen).
Wir verwenden weiterhin $p,q,r$ für atomare Terme.
Die $K_ip$ bedeutet das Agent $i$ die Tatsache $p$ weis.
Hier nun eine komplexeres Beispiel einer Formel in $KT45^n$ : $K_1p \wedge K_1 \neg K_2 K_1 p$.
Sie besagt: Agent 1 weis $p$, außerdem weis Agent 1, das Agent 2 nicht weis, dass er $p$ weis.
Mit $G$ beschreiben wir eine Gruppe von Agenten $G= \{1,2,3,\dots,n\}$.
Will man nun Aussagen das eine Gruppe $G$ von Agenten einen Umstand $p$ weis : $K_1 p \wedge K_2 p \wedge \dots K_n p$ benutzt man dafür die Formulierung $E_G p$.
Es gelden die selben Bindungsstärke der Operatoren wie in der Auflistung \Ref{bindungsstaerke}.
Wobei $K_i$ wie der $\Box$-Operator behandelt wird.
\citeHuth{S.335ff}


Bei erster Betrachtung mag man annehmen, dass eine Tatsache $\phi$ nicht bekannter sein kann als $E_G \phi$.
$E_G E_G \phi$ stellt jedoch mehr Wissen dar als $E_G \phi$, denn es besagt nicht nur das jeder etwas weis, sondern auch das jeder weis, dass es jeder weis.
Genauso stellt $E_G E_G E_G \phi$ wiederum noch mehr Wissen dar, denn jeder Weis, dass alle etwas wissen und \emph{das} ist wiederum bekannt.
Dies lässt sich ins unendliche fortsetzen $E_G E_G \dots \phi$.
Da es aber nur möglich ist finite Aussagen zu machen, für diesen Umstand des Allgemeinwissens ein extra Operator $C_G$ eingeführt und über seine Semantik definiert.
Wir bezeichnen als mit $C_G$ das Allgemeinwissen innerhalb einer Gruppe $G$.
Mit $D_G$ wollen wir verteiltes Wissen beschreiben.
Verteiltes wissen ist dem Einzelnen evtl. nicht bekannt, kann jedoch gefolgert werden, sobald alle Beteiligten der Gruppe ihr Wissen vereinen.
Die Buchstaben $C$ und $D$ stammen aus dem Englischen für common-knowlage und distributed-knowlage.

\textbf{multi agent systeme}
\begin{definition}
	\label{def:bnf_kt45n}
	Eine Formel $\psi$ der multi modal Logik $KT45^n$ ist definiert durch folgende Grammatik:
	\begin{equation}
		\label{eqn:bnf_kt45n}
		\phi ::= \bot|\top|p|(\neg\phi)|(\phi\wedge\phi)|(\phi\vee\phi)|(\phi\rightarrow\phi)|
		(\phi\leftrightarrow\phi)|(K_i\psi)|(E_G\psi)|(C_G\psi)|(D_G\psi)
	\end{equation}
	wobei $p$ irgendeine atomare Formel ist und $i \in \Fancy{A}$ sowie $G \subseteq \Fancy{A}$ gilt.
	$E_\Fancy{A}, C_\Fancy{A}, D_\Fancy{A}$ werden zur Einfachheit ohne den extra Index geschrieben $E,C,D$.
	\citeHuth{S.335f}
\end{definition}


Vergleicht man diese Definition mit der von 
\Def{syntax} so stellt man fest, dass anstelle des 
$\Box$ Operator nun eine Vielzahl von Operatoren gibt: 
$K_i, E_G, C_G, D_G$ für alle $ G \subseteq \Fancy{A}$.
Im Folgenden wird gezeigt das sich diese Operatoren $\Box$ ähnlich verhalten.
Es gibt kein explizites Analogon zu $\Diamond$.
Es ist aber entsprechend gleichbedeutend mit $\neg K_i \neg, \neg E_G \neg, \neg C_G \neg, \neg D_G \neg$.


\begin{definition}
	Ein Model \MMModelDef der \MML $KT45^n$ mit der Menge $\Fancy{A}$ von $n$ Agenten wird beschrieben durch drei Dinge:
	\begin{enumerate}
		\item einer Menge $W$ von möglichen Welten
		\item für jedes $i \in \Fancy{A}$, der Gleichheitsrelation $R_i$ auf $W$ $R_i \subseteq W \times W)$ auch Erreichbarkeitsrealtion gennant und
		\item der Labeling-Funktion $L: W \rightarrow \Fancy{P}(Atome)$
	\end{enumerate}
	\cite[S.336f]{huth2004logic}
\end{definition}

Vergleicht man diese Definition mit der \Def{model}, aus der \ML so lässt sich folgendes feststellen.
Es sind nur mehrere Relationen $R_i$, eine für jeden Agenten $i$ definiert.
Außerdem wird vorausgesetzt, dass $R$ eine Gleichheitsrelation ist, also reflexiv und symmetrisch.\\
Dieser Umstand wird bei der graphischen Darstellung von $KT45^n$ Modellen ausgenutzt.
So sind die Kanten mit den Relationen beschriftet für die sie gelten, außerdem gibt es keine Notwendigkeit für Pfeile, weil eine Verbindung durch die Symmetrie-Eigenschaft immer in beide Richtungen besteht.
Streng genommen müsste auch jede Welt, aufgrund der Transitivität, eine Verbindung auf sich selbst haben, die in jeder Relation definiert ist.
Weil dieser Umstand stand aber für alle Welten außnahmslos gilt, wird zugunsten der Übersichtlichkeit auf die Verbindungen verzichtet. \todo{Abbildung verlinken}

\todo{Bsp: grphic aus huth S. 336 erstellen}



\begin{definition}
		Gegeben ein Model $\mathds{M} = (W,(R_i)_{i \in \mathds{A}}, L)$ der $KT45^n$ und eine Welt $w \in W$, so definieren $\psi$ als \emph{Wahr} durch die Erfüllung der Relation $x \vDash \psi$ durch folgende Regeln:
		\begin{align}
			x &\Vdash p\text{ gdw. }p \in L(x)\\
			x &\Vdash \neg \phi\text{ gdw. }x \nVdash \phi\\
			%
			x &\Vdash \phi \wedge \psi\text{ gdw. }x \Vdash \phi\text{ und } x \Vdash \psi\\
			x &\Vdash \phi \vee \psi\text{ gdw. }x \Vdash \phi \text{, oder } x \Vdash \psi\\
			%
			x &\Vdash \phi \rightarrow \psi\text{ gdw. }x \Vdash \psi\text{, immer wenn gilt }x \Vdash \phi\\
			%
			x &\Vdash K_i\psi \text{ gdw. } \forall y \in W, R_i(x,y) \text{, } y \Vdash \psi \text{ impliziert}\\
			x &\Vdash E_G\psi \text{ gdw. } \forall i \in G, x \Vdash K_i\psi\\
			x &\Vdash C_G\psi \text{ gdw. } \forall k \geq 1 \text{, und es gilt } x \Vdash E^k_G\psi \text{.} \text{Wobei } E^k_G \text{ meint } E_{G}E_{G}\dots E_{G} \text{ k-mal.}\\
			x &\Vdash D_G\psi \text{ gdw. } \forall y \in W, y \Vdash \psi \text{gilt, immer wenn auch } R_i(x,y), \forall i \in G \text{gilt.}\\
		\end{align}
		\cite[S.337]{huth2004logic}
\end{definition}

Wir wollen wieder diese Definition der \MML mit ihrem Analogon in der \ML (\Def{reasoning}) vergleichen.
Die typischen Boolean Operatoren sind identisch definiert.
Alle $K_i$ verhalten sich wie der $\Box$ Operator, nur jeweils bezogen auf ihre Relation $R_i$.
$E_G$ ist auf Basis von $K$ definiert und $C_G$ wiederum auf Basis von $E_G$.\\
Es gibt keien Definition für $\Diamond$ weil dies über $\neg K \neg$ ausgedrückt werden kann.

Viele der festgestellen Eingenschaften der \ML gelten auch in der \MML nur jeweils mit Bezug auf die entsprechende Relation $R_i$:
\begin{enumerate}
	\item Ein \textbf{Frame} \MMFrameDef besteht aus einer Menge von Welten $W$ und einer Gleichheitsrelation $R_i$ für jedes $i \in \Fancy{A}$.
	\item Ein Frame \MMFrameDef erfüllt eine Formel $\phi$ gdw. für jede Labeling-Funktion \LabelFuncDef in jeder Welt $w \in W$, $\Fancy{M}, w \vDash \phi$ gilt, mit \MMModelDef. Dann schreiben wir $\Fancy{F} \vDash \phi$.
\end{enumerate}

Das \Theo{gErreichbarkeit} ist nützlich wenn es um die Beantwortung von Formel geht, die $E$ oder $C$ enthalten.
Wir wollen nun den Begriff der G-Erreichbarkeit erklären.
Seit \MMModelDef ein Model für $KT45^n$ und $x,y \in W$.
Wir nennen $y$ G-erreichbar in $k$ Schritten von $x$ wenn es $w_1, w_2, \dots, w_{k-1} \in W$ und $i_1, i_2, \dots, i_k$ in $G$ gibt, sodass 

\begin{equation*}
	x R_{i_1} w_1 R_{i_2} w_2 \dots R_{i_{k-1}} w_{k-1} R_{i_k} y
\end{equation*}

gilt.
Die Formulierung meint $R_{i_1}(x,w_1)m, R_{i_2}(w_1,w_2), \dots, R_{i_k}(w_k,w_y)$.
Eine Welt $y$ wird einfach nur G-erreichbar von $x$ genannt, wenn diese durch eine feste Anzahl an Schritten $k$ von $x$ G-erreichbar in $k$ Schritten ist.

\begin{theorem}
	\label{theo:gErreichbarkeit}
	\begin{enumerate}
		\item $x \Vdash E_{G}^{k} \phi$, gdw. für alle $y$, die in $k$ Schritten von $x$ G-erreichbar sind, $y \Vdash \phi$ gilt.
		\item $x \Vdash C_{G} \phi$, gdw. für alle $y$, die von $x$ G-erreichbar sind, $y \Vdash \phi$ gilt.
	\end{enumerate}
\end{theorem}

\note{Beweis führen?}


\paragraph{Valide Formeln in $KT45^n$}
\label{par:valid_formulas_in_kt45n}
Das Schema $K$ gilt für alle Operatoren.
Alle Ebenen des Wissens sind also abgeschlossen gegenüber der logischen Konsequentz.
Wenn z.B. eine Zusammenhang $\phi$ Allgemeinwissen ist dann sind auch alle logischen Folgerungen daraus wieder Teil des Allgemeinwissens.\\
Die Operatoren $E,C,D$ sind boxähnlich weil sie Universallquantoren über die Relationen $R_{E_G}, R_{D_G}, R_{C_G}$ sind.


\begin{align}
	R_{E_G}(x,y) &\text{ gdw. } R_i(x,y) \text{ für einige } i \in G\\
	R_{D_G}(x,y) &\text{ gdw. } R_i(x,y) \text{ für alle } i \in G\\
	R_{C_G}(x,y) &\text{ gdw. } R_{E_G}^k(x,y) \text{ für jedes } k \geq 1
\end{align}


Daraus folgt, dass $E_G, D_G, C_G$ das Schema $K$ im Bezug auf $R_{E_G}, R_{D_G}, R_{C_G}$ erfüllen.
Was ist mit den anderen Schemata $T,4,5$?\\
Da $R_i$ eine Gleichheitsrelation ist folgt nach \Theo{aehnlichkeitstheorie} und  \Tab{attributesIncludingR} das für jedes $K_i$ gilt:
\begin{itemize}
	\item $\Four{K_i}{\phi}$ \emph{positive Introspektion}
	\item $\Five{K_i}{\phi}$ \emph{negative Introspektion}
	\item $\Truth{K_i}{\phi}$ \emph{Wahrheit}
\end{itemize}

$R_{D_G}$ erfüllt die die Schemata $T,4,5$, weil sie ebenfalls eine Gleichheitsrelation ist.
Die Schemata gelten aber nicht automatisch für $E_G$ und $E_C$.
$\Four{E_G}{\phi}$ gilt z.B. nicht sonst wäre Allgemeinwissen lediglich, das was jeder weiß.
$\Five{E_G}{\phi}$ gilt ebenfalls nicht.\\
Die Ursache für diese Sachverhalte ist die Tatsache, dass $R_{E_G}$ ist nicht notwendigerweise eine Gleichheitsrelation ist, obwohl dies für jedes $R_i$ gilt.\\
$\Truth{E_G}{\phi}$ ist hingegen der Fall, weil $R_{E_G}$ reflexiv ist.
Für den Fall $G \neq \emptyset$ ist $E_G \phi$ immer leer, auch wenn $\phi = \bot$, sprich $E_G \bot \rightarrow \emptyset$. \todo{kann man das so schreiben?}\\
$R_{C_{G}}$ ist eine Gleichheitsrelation, daraus folgt, dass $T,4,5$ gelten, wobei 5 $G \neq \emptyset$ fordert.





\subsection{\ND in $KT45^n$}
Die \ND von $KT45$ wird erweitert um neue Arten von blauen Boxen.
Jeweils eine für jeder der entsprechenden Operatoren.
Der Operator D wird hier nicht behandelt.\\
Wie in \Abs{par:valid_formulas_in_kt45n} gesehen könne die Axiome T 4 5 für jedes $K_i$ eingesetzte werden. 4 und 5 können hingegen für $C_G$ aber nicht für $E_G$ eigenutzt.\\
Die Regeln CE und CK sind genau genommen eine Menge von Regeln für jede Wahl eines bestimmten $k$, der Einfachheit halber werden sie aber nur mit CE und CK bezeichnet.
$EK_i$ verhält sich wie eine generelle \emph{und}-Elimination und $KE$ wie eine generelle \emph{und}-Einführung.\\
Wie auch bei den Regeln zur KT45 kann man sich die Regeln K4, K5, C4 und C5 als eine Art Striktheitsentschärfung für das Importieren und Exportieren von Formeln in bestimme Boxen vorstellen.\\
Weil K4 es erlaubt um jedes $K_i$ ein weiters $K_i$ zu ergänzen erlaubt es effektiv das unveränderte Importieren von $K_i \phi$ Formeln in $K_i$ Boxen.
Ähnliches gilt für C5 das es uns erlaubt $\neg C_G$ Formeln in $C_G$ Boxen zu importieren.\\
Das Öffnen einer Box kann man sich intuitiv als das folgern auf Basis des Wissens des entsprechenden Agenten vorstellen.
Unter dieser Betrachtung ist es ebenfalls intuitiv einleuchtend, dass eine Tatsache $\phi$ nicht einfach in eine Box gebracht werden kann, denn die existent der Tatsache bedeutet nicht gleichzeitig, dass der entsprechende Agent diese auch weis.\\
Wir erinnern uns, dass ein Agent nicht falsches wissen kann.
Daher gilt besondere Sorgfalt mit der Regel $\neg i$.
Sie darf nicht angewendet werden, wenn eine der verwendeten Annahmen außerhalb der Box existiert.\\
$C\phi$ ist besonders mächtig, weil es erlaubt die Formel $\phi$ in jeder Box zu verwenden unabhängig von dessen Schachtelungstiefe.
Die Regel $E^k \phi$ hingegen erlaub die Verwendung von $\phi$ nur in Boxen der Schachtelungstiefe $\leq k$.\citeHuth{S.339ff}





\subsection{Formalisierung des Wise Men Rätsels in $KT45^n$}
Da wir nun eine Definition der Logik $KT45^n$ könnnen wir das Wise Men Rätsel darin formulieren und lösen.\\
Zur Erinnerung: Der König setzt jeder der drei weisen Männen einen Hut auf, sodass jeder die Hüte der anderen nicht jedoch seinen eigenen sehen kann.
Es ist allgemein bekannt das es nur 3 rote und 2 weiße Hüte gibt.
Der König fragt nun jeden der Männer der reihe nach welche Farbe der Hut hat, den der Man trägt.
Wir gehen davon aus, das sowohl der erste als auch der zwei weise Mann die Aussage machen, dass sie es nicht wissen.
Es soll nun mithilfe der $KT45^n$ formal gezeigt, dass der dritte Mann dann zwangsläufig weiß, dass er einen roten Hut trägt.

Im Folgenden wird $p_i$ notieren, dass der Mann $i$ einen roten Hut auf hat und 
$\neg p_i$, dass der Mann $i$ einen weißen Hut trägt.

Die Menge $\Gamma$ enthält Formeln, die das Allgemeinwissen in $KT45^n$ beschreiben:
\begin{subequations}
	\begin{align}
		\{&C(p_1 \vee p_2 \vee p_3) \label{eq:min_ein_roter_hut}\\
		\label{eq:zweiter_kennt_farbe_von_erster}
		&C(p_1 \rightarrow K_2 p_1), C(\neg p_1 \rightarrow K_2 \neg p_1),\\
		&C(p_1 \rightarrow K_3 p_1), C(\neg p_1 \rightarrow K_3 \neg p_1),\\
		&C(p_2 \rightarrow K_1 p_2), C(\neg p_2 \rightarrow K_1 \neg p_2),\\
		&C(p_2 \rightarrow K_3 p_2), C(\neg p_2 \rightarrow K_3 \neg p_2),\\
		&C(p_3 \rightarrow K_1 p_3), C(\neg p_3 \rightarrow K_1 \neg p_3),\\
		\label{eq:zweiter_kennt_farbe_von_letzter}
		&C(p_3 \rightarrow K_2 p_3), C(\neg p_3 \rightarrow K_2 \neg p_3)\}
	\end{align}
\end{subequations}

\Eq{min_ein_roter_hut} beschreibt den Umstand, dass mindestens einer der Männer einen roten Hut tragen muss und damit auch das es nicht mehr als zwei weiße geben kann.
Die \Eq{zweiter_kennt_farbe_von_erster} bis \Eq{zweiter_kennt_farbe_von_letzter} beschreiben, dass jeder die Hutfarbe des anderen weiß, jedoch nicht seine eigenen und das dies allgemein bekannt ist.
Es ist zu beachten, dass jeweils der positiv wie der negativ Fall festgehalten wird.

Die Aussage des ersten Mannes, dass er nicht weis welche Farbe sein Hut hat lässt sich wie folgt formalisieren:
\begin{equation*}
	C(\neg K_1 p_1 \wedge \neg K_1 \neg p_1)
\end{equation*}

entsprechend die Aussage des zweiten Mannes:
\begin{equation*}
	C(\neg K_2 p_2 \wedge \neg K_2 \neg p_2)
\end{equation*}

Ein naiver Ansatz das Rätsel zu lösen könnte so aussehen:

\begin{equation}
	\Gamma, C(\neg K_1 p_1 \wedge \neg K_1 \neg p_1), C(\neg K_2 p_2 \wedge \neg K_2 \neg p_2) \Vdash K_3 p_3
\end{equation}

Auch wenn $KT45$ ist eine komplexe Logik, reicht jedoch noch nicht aus, um das Beispiel in einem Schritt erklären zu können.
Es fehlt der temporale (zeitliche) Aspekt.
$C \neg K_1 p_1$ mag z.B. wahr sein muss es aber nicht bleiben.
Nach der Aussage des Erste ist z.B. $C p_1$ bekannt und $C \neg K_1 p_1$ damit hinfällig.

Wissen hat, in unserem idealisierten System, die Eigenschaft sich anzuhäufen und bestehen zu bleiben. 
\emph{Unwissenheit hat diese Eigenschaften nicht.}
Negative Wissensaussagen der form $\neg K \phi$ können also im zeitlichen Verlauf ihre Gültigkeit verlieren.
Erneut: $KT45^n$ kennt keine zeitlichen Zusammenhänge.

Um den Beweis des wise men Rätsels zu führen, der einen zeitlichen aspekt fordert, werden wir uns durch Schrittweise beweisen behelfen indem wir jeweils den Wissensstand vor einer Ankündigung betrachten.
Die Lösung des wise men Rätsels benötigt folglich zwei Schritte.

Informel konnte die Kombination RWW nach der ersten Ankündigung ausgeschlossen werden.
Damit wird also $p_1 \vee p_2 \vee p_3$ zu $p_2 \vee p_3$.\\
Im ersten Schritt soll genau dies nun aus dem anfänglichen Wissen gefolgert werden:
$\Gamma, C(\neg K_1 p_1 \wedge \neg K_1 \neg p_1), \Vdash C(p_2 \vee p_3)$
weil $p_2 \vee p_3$ eine positive Formulierung ist, bleibt sie auch in späteren Schnitten gültig.\\
Schritt zwei kann nun das in Schritt~1 gefolgerte Wissen nutzen um die Aussage das der dritte Mann weis, dass er einen roten Hut trägt zu folgern:
$\Gamma, C(p_2 \vee p_3), C(\neg K_1 p_1 \wedge \neg K_1 \neg p_1), \Vdash K_3 p_3$

Bei dieser Vorgehensweise handelt es sich um eine schwierige Methode, weil für jeden Schritt \emph{positives} Wissen gefunden werden muss, dass zusätzlich auch noch zielführend ist für den eigentlichen zu beweisenden Schluss.

\todo{kompletten Beweis aufführen}
\todo{kompletten Beweis diskutieren}








% subsection the_model_logic_kt45_n_ (end)


% section normal_modal_logic (end)