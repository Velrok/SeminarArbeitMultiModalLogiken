%!TEX root = /Users/velrok/Dropbox/TheoInf Seminar/Ausarbeitung/Main.tex


\section{Multi-Modal-Logic} % (fold)
\label{sec:multi_modal_logic}
Multi Modal Logiken sind Modal-Logiken die mehr als eine Modularität der Wahrheit enthalten.
Ein Beispiel dafür sind Multi-Agent-Systeme. In einem solchen System, kann ein Agent nicht nur Folgerungen auf Basis seines Wissens, sonder auch auf Basis des Wissen über das Wissen der anderen anstellen. Also Aussagen der Art: Da ich weis das er A weis kann ich B folgern.
Konkret wird dieses Kapitel die Logik $KT45^n$, den allgemeinen Fall der Wissenslogik $KT45$, anhand der klassischen Logik Rätsel \emph{Wise-Men} und \emph{Muddy-Children} darstellen.

\subsection{Das Wise-Men Rätsel} % (fold)
\label{sub:das_wise_men_raetsel}

% subsection das_wise_men_ (end)

\subsection{Das Muddy-Children Rätsel} % (fold)
\label{sub:das_muddy_children_raetsel}

% section das_muddy_children_rätsel (end)

\subsection{Die Modal-Logik $KT45^n$ (Multi-Agent-Wissen)} % (fold)
\label{sub:the_modal_logic_kt45_n_}

% subsection the_model_logic_kt45_n_ (end)

% section normal_modal_logic (end)