%!TEX root = /Users/velrok/Dropbox/TheoInf Seminar/Ausarbeitung/Main.tex

\chapter{Eigenschaften, Anwendungsfelder} % (fold)
\label{sec:eigenschaften_anwendungsfelder}

\paragraph{Wahrheits-Modi} % (fold)
\label{par:wahrheits_modi}

Die Aussagenlogik und Prädikatenlogik kennen nur eine Art von Wahr oder Falsch. Im realen Leben unterscheiden wir jedoch ganz intuitiv zwischen einer Vielzahl von unterschiedlichen Wahrheiten.
Die Aussage \emph{"Frau Merkel ist die Bundeskanzlerin der BRD"} ist \textbf{im Moment} wahr es kann sich jedoch bei der nächsten Wahl ändern.
Die Aussage \emph{"Die Erde hat einen Mond"} ist jetzt wahr und wird mit an Sicherheit grenzender Wahrscheinlichkeit auch in Zukunft wahr sein. 
Sie ist aber nicht notwendiger Weise Wahr, denn es hätten ja auch 2 oder 3 Monde sein können.
Ein Beispiel für eine notwendigerweise wahre Aussage ist \emph{''ein Junggeselle ist unverheiratet''}. Wir können uns keinen verheirateten Junggestellen vorstellen.\todo{Zitat einfügen}

% paragraph wahrheits_modi (end)

\paragraph{Anwendungsfelder} % (fold)
\label{par:anwendungsfelder}

In der Informatik ist das Schlussfolgern über verschiedene Arten der Wahrheit nützlich in Bereichen wie \emph{Model Checking} und \emph{AI (Artificial Intelligence)}. \\
Im Model Checking kommen vor allem Temporal Logiken zum Einsatz, um Wahrheitsaussagen zu unterschiedlichen Zeiten im Programmablauf treffen zu können. \\
Im Bereich der AI, werden Multi-Modal-Logiken in Multi-Agent-Systemen verwendet. 
Solche Systeme sind in der Lage Schlussfolgerungen, nicht nur aus dem eigenen Wissen, sondern auch aus dem Wissen über das Wissen anderer und deren Wissen zu ziehen.\\
Einfache Logiken modellieren nur eine Modularität von Wahrheit, z.B. notwendigerweise wahr, komplizierte Logiken modellieren auch mehrere, z.B. wahr nach allem was Agent $i$ weis, für $ 0 < i < k$. \cite[S.306f]{huth2004logic}\\

% paragraph anwendungsfelder (end)

\paragraph{Struktur der Arbeit} % (fold)
\label{par:struktur_der_arbeit}

Diese Arbeit ist in zwei Bestandteile aufgeteilt. 

Kapitel \ref{sec:modal_logic} beschäftigt sich mit dem Grundlagen von Modal-Logiken. 
Es erklärt Syntax, Semantik die Modellierung in Form von Kripke-Struckturen und die possible Word Semantik. \todo{stimmt das so?}
Es werden wichtige Eigenschaften von Modal-Logiken aufgeführt und erklärt, sowie der Zusammenhang zwischen der Relation $R$ im Kipke-Model und den vorher beschriebenen Eigenschaften verdeutlicht, bekannt als \emph{Ähnlichkeitstheorie}.
Am Ende wird das Thema anhand der Modal-Logik $KT45$ (für Wissen) konkretisiert.

Kapitel \ref{sec:multi_modal_logic} erklärt Multi-Modal-Logiken, als Modal-Logiken die mehr als nur eine Modalität vereinen. Das Kapitel beginnt mit der Darstellung der \emph{the wise men} und \emph{muddy children} Rätsel. 
Anhand dieser Beispiele wird die Verallgemeinerung der Modal-Logik $KT45$ verdeutlicht und zur Anwendung gebracht um diese Rätsel formal zu lösen.
Im Zuge dieser Lösung wird genauer auf die Multi-Modal-Logik $KT45^n$ eingegangen und deren Unterschiede und Erweiterungen im Vergleich zur Logik $KT45$ erklärt.




% paragraph struktur_der_arbeit (end)

% section eigenschaften_abgrenzung_zur_aussagenlogik (end)
