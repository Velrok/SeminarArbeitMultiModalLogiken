%!TEX root = /Users/velrok/Dropbox/TheoInf Seminar/Ausarbeitung/Main.tex

\chapter{Eigenschaften, Anwendungsfelder} % (fold)
\label{sec:eigenschaften_anwendungsfelder}

\paragraph{Wahrheits-Modi} % (fold)
\label{par:wahrheits_modi}

Sowohl die Aussagenlogik als auch die Prädikatenlogik kennen nur eine Art von Wahr oder Falsch. 
Im realen Leben unterscheiden wir jedoch ganz intuitiv zwischen einer Vielzahl von unterschiedlichen Wahrheiten.
Die Aussage \emph{"Frau Merkel ist die Bundeskanzlerin der BRD"} ist \textbf{im Moment} wahr, kann sich jedoch bei der nächsten Wahl ändern.
Die Aussage \emph{"Die Erde hat einen Mond"} ist jetzt wahr und wird mit an Sicherheit grenzender Wahrscheinlichkeit auch in Zukunft wahr sein. 
Sie ist aber nicht notwendigerweise Wahr, denn es hätten ja auch 2 oder 3 Monde sein können.
Ein Beispiel für eine notwendigerweise wahre Aussage ist \emph{''ein Junggeselle ist unverheiratet''}. 
Den ein Junggestelle ist per Definition unverheiratet; es ist genau diese Eigenschaft, die ihn zum Junggestellen macht \vglHuth{S. 306}.

% paragraph wahrheits_modi (end)

\paragraph{Anwendungsfelder} % (fold)
\label{par:anwendungsfelder}

In der Informatik ist das Schlussfolgern über verschiedene Arten der Wahrheit nützlich in Bereichen wie \emph{Model Checking} und \emph{AI (Artificial Intelligence)}. \\
Im Model Checking kommen vor allem Temporal-Logiken zum Einsatz, um Wahrheitsaussagen zu unterschiedlichen Zeiten im Programmablauf, treffen zu können. \\
Im Bereich der AI, werden \MMLn in Multi-Agent-Systemen verwendet. 
In solchen Systemen sind die Agenten in der Lage Schlussfolgerungen nicht nur aus dem eigenen Wissen, sondern auch aus dem Wissen über das Wissen anderer Agenten zu ziehen.\\
Einfache Logiken modellieren nur eine Modalität von Wahrheit, z.B. notwendigerweise wahr. Komplizierte Logiken modellieren auch mehrere Arten von Wahrheit, z.B. wahr nach allem was Agent $i$ weiß, für $ 0 < i < k$ \vglHuth{S.306f}.\\

% paragraph anwendungsfelder (end)

\paragraph{Struktur der Arbeit} % (fold)
\label{par:struktur_der_arbeit}

Diese Arbeit ist in zwei Bestandteile aufgeteilt. 

Kapitel \ref{sec:modal_logic} beschäftigt sich mit den Grundlagen von \ML. 
Es erklärt Syntax, Semantik, die Modellierung in Form von \KSn und die \PW Semantik.\\
Es werden wichtige Eigenschaften der \ML aufgeführt und erklärt sowie der Zusammenhang zwischen der Relation $R$ im \KM und den vorher beschriebenen Eigenschaften verdeutlicht. 
Diese Zusammenhänge sind bekannt als \emph{Ähnlichkeitstheorie}.
Am Ende wird das Thema anhand der \ML $KT45$ (für Wissen) konkretisiert.

Kapitel \ref{sec:multi_modal_logic} erklärt \MMLn, als Logiken die mehr als nur eine Modalität vereinen.
Das Kapitel beginnt mit der Darstellung des \WMR. 
Anhand dieses Beispiels wird die Verallgemeinerung der \ML $KT45$ verdeutlicht und zur Anwendung gebracht um das Rätsel formal zu lösen.
Im Zuge dieser Lösung wird genauer auf die \MML $KT45^n$ eingegangen und deren Unterschiede und Erweiterungen im Vergleich zur Logik $KT45$ erklärt.




% paragraph struktur_der_arbeit (end)

% section eigenschaften_abgrenzung_zur_aussagenlogik (end)
