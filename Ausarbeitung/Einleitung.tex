%!TEX root = /Users/velrok/Dropbox/TheoInf Seminar/Ausarbeitung/Main.tex

\section{Eigenschaften, Anwendungsfelder} % (fold)
\label{sec:eigenschaften_anwendungsfelder}

\paragraph{Wahrheits-Modi} % (fold)
\label{par:wahrheits_modi}

Die Aussagenlogik und Prädikatenlogik kennen nur eine Art von Wahr oder Falsch. Im realen Leben unterscheiden wir jedoch ganz intuitiv zwischen einer Vielzahl von unterschiedlichen Wahrheiten.
Die Aussage \emph{"Frau Merkel ist die Bundeskanzlerin der BRD"} ist \textbf{im Moment} wahr es kann sich jedoch bei der nächsten Wahl ändern.
Die Aussage \emph{"Die Erde hat einen Mond"} ist jetzt wahr und wird mit an Sicherheit grenzender Wahrscheinlichkeit auch in Zukunft wahr sein. Sie ist aber nicht notwendiger Weise Wahr, in dem sinne als es hätten auch 2 oder 3 Monde sein können.
Ein Beispiel für eine notwendigerweise wahre Aussage ist \emph{"ein Junggeselle ist unverheiratet"}. Wir können uns keinen verheirateten Junggestellen vorstellen.<Zitate notwendig>

% paragraph wahrheits_modi (end)

\paragraph{Anwendungsfelder} % (fold)
\label{par:anwendungsfelder}

In der Informatik ist das Schlussfolgern über verschiedene Arten der Wahrheit nützlich in den Bereichen wie \emph{Model Checking} und \emph{AT (AI (Artificial Intelligence)}. \\
Im Model Checking kommen vor allem Temporal Logiken, ein spezial Fall der Modal Logik, zum Einsatz, um Wahrheitsaussagen zu unterschiedlichen Zeiten im Programmablauf treffen zu können. \\
Im Bereich der AI, werden Multi-Modal-Logiken in Multi-Agent-Systemen verwendet. Solche Systeme sind in der Lage Schlussfolgerungen, nicht nur aus dem eigenen Wissen, sondern auch aus dem Wissen über das Wissen anderer und deren Wissen, zu ziehen.\\
Einfache Logiken modellieren nur eine Modularität von Wahrheit, z.B. notwendigerweise wahr, komplizierte Logiken modellieren auch mehrere, z.B. wahr nach allem was Agent $i$ weis, für $ 0 < i < k$. \cite[S.306f]{huth2004logic}\\

% paragraph anwendungsfelder (end)



% section eigenschaften_abgrenzung_zur_aussagenlogik (end)
