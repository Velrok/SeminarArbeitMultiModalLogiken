%!TEX root = /Users/velrok/Dropbox/TheoInf Seminar/Ausarbeitung/Main.tex

\section{Syntax} % (fold)
\label{sec:syntax}
Die Syntax der Modal Logik entspricht der der Aussagenlogik mit den Erweiterungen $\square$ und $\Diamond$. 
Wie die Negation sind diese unär, das heißt sie beziehen sich nur auf die ihr folgende Formel. Im Folgenden werden die Zeichen $p, q, r, p_3$ für atomare Formeln verwendet \vglHuth{S. 307f}.\\
\\
\begin{definition}
	\label{def:syntax}
	Die folgende BNF (Backus-Naur-Form) beschreibt die Syntax der möglichen multi modal Formeln $\phi$.

	\begin{equation}
		\label{eqn:bnf}
		\phi ::= \bot|\top|p|(\neg\phi)|(\phi\wedge\phi)|(\phi\vee\phi)|(\phi\rightarrow\phi)|
		(\phi\leftrightarrow\phi)|(\square\phi)|(\Diamond\phi)
	\end{equation}
	\citeHuth{S.307}
\end{definition}

Die Formeln 
\begin{align}
	(p \wedge \Diamond(p \rightarrow \square \neg r))
	\square((\Diamond q \wedge \neg r) \rightarrow \square p )	
\end{align} 

sind Beispiele für syntaktisch korrekte \MML Formeln. 
Wie auch bei der \AL binden die unären Operatoren stärker als die Binären, sodass unnötige Klammern weggelassen werden können, um die Leserlichkeit zu verbessern.


Die folgende Liste sortieren die Operatoren nach ihrer Bindungsstärke, beginnend mit den am stärksten bindenden:\\
\label{bindungsstaerke}
\begin{itemize}
	\item $\neg, \square, \Diamond$
	\item $\wedge, \vee$
	\item $\rightarrow, \leftrightarrow$
\end{itemize}

Im allgemeinen werden die Symbole $\square$,als Box, und $\Diamond$ als Raute gelesen. 
Spezifiziert man eine konkrete Logik so werden diese entsprechend ihrer Interpretation gelesen. 
In der Logik für Notwendigkeit wird $\square$ als notwendig und $\Diamond$ als möglich gelesen. 
In Logik für über das Wissen eines Agenten Q, wird $\square$ als Q weiß und $\Diamond$ als soweit Q weiß, gelesen.


% section syntax (end)
